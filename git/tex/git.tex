\documentclass{article}
\usepackage[margin=1in]{geometry}
\usepackage{amsfonts,amsmath,amssymb}
\usepackage{graphicx}
\usepackage{tcolorbox}
\usepackage{hyperref}
\usepackage{fancyhdr}
\usepackage{float}
\usepackage{enumitem}
\usepackage{titlesec}
\usepackage{lmodern}
\usepackage{xcolor}

% Colori personalizzati
\definecolor{primary}{HTML}{2E86C1} % Blu moderno
\definecolor{secondary}{HTML}{117864} % Verde scuro
\definecolor{accent}{HTML}{D35400} % Arancione acceso
\definecolor{lightgray}{HTML}{F2F3F4} % Grigio chiaro per sfondi

% Impostazioni header e footer
\pagestyle{fancy}
\fancyhf{}
\fancyhead[L]{\textbf{\textcolor{primary}{Git}}}
\fancyhead[R]{\textcolor{secondary}{Valerio Molinari}}
\fancyfoot[C]{\thepage}

% Hyperlink
\hypersetup{
    colorlinks=true,
    linkcolor=primary,
    urlcolor=accent,
    citecolor=secondary
}

% Titoli stilizzati
\titleformat{\section}{\color{primary}\normalfont\Large\bfseries}{}{0em}{}
\titleformat{\subsection}{\color{secondary}\normalfont\large\bfseries}{}{0em}{}
\titleformat{\subsubsection}{\color{accent}\normalfont\normalsize\bfseries}{}{0em}{}

% Evidenziazione con box
\newtcolorbox{highlight}{colback=lightgray,colframe=primary!80!black,boxrule=0.5mm,arc=4mm,top=2mm,bottom=2mm,left=4mm,right=4mm}

% Spaziatura e interlinea
\setlength{\parskip}{1em}
\setlength{\parindent}{0pt}
\renewcommand{\baselinestretch}{1.5}

% Nuove definizioni
\newcommand{\important}[1]{\textcolor{accent}{\textbf{#1}}}

\begin{document}

% Copertina
\begin{titlepage}
\begin{center}
\vspace*{3cm}
\Huge\textcolor{primary}{\textbf{Formazione VIM}} \\[1cm]
\Large\textcolor{secondary}{Appunti di Git} \\[1cm]
\textcolor{accent}{Valerio Molinari}\\
\vfill
\today
\end{center}
\end{titlepage}

% Indice
\tableofcontents
\newpage

% Esempio di sezione
\section{Introduzione}
Git è uno strumento di controllo di versione distribuito, 
utilizzato per tenere traccia delle modifiche ai file e 
coordinare il lavoro tra più persone.

La sincronizzazione si basa sull'utilizzo di \textbf{istantanee} 
(\textit{snapshot}) dei file, che vengono salvate in un
\textbf{repository}.

% \begin{highlight}
% Questa è un'area evidenziata per definizioni importanti o esempi pratici.
% \end{highlight}

\subsection{Funzionamento}
\begin{enumerate}
    \item \textbf{Working Directory}: è la cartella in cui si sta lavorando.
    \item \textbf{Staging Area}: è l'area in cui vengono preparate le modifiche da salvare.
    \item \textbf{Repository}: è il database in cui vengono salvate le istantanee.
\end{enumerate}

Un'istantanea è composta dai file che sono stati modificati rispetto
alla precedente e da puntatori ai file invariati.

La repository si trova all'interno della cartella .git, che contiene
tutte le informazioni necessarie per il controllo di versione.


\section{Git Bash}
Su Windows, Git Bash è una shell che permette di utilizzare i comandi
di Git da riga di comando. È possibile aprirla cliccando con il tasto
destro del mouse in una cartella e selezionando \textit{Git Bash Here}.
\begin{highlight}
    I comandi di Git Bash sono simili a quelli di Linux.
\end{highlight}

\subsection{Comandi}
\begin{itemize}
    \item \texttt{ll}: elenca i file e le cartelle.
    % \item \texttt{cd <cartella>}: sposta nella cartella specificata.
    % \item \texttt{pwd}: mostra il percorso della cartella attuale.
    % \item \texttt{mkdir <cartella>}: crea una nuova cartella.
    % \item \texttt{touch <file>}: crea un nuovo file.
    % \item \texttt{rm <file>}: rimuove un file.
    % \item \texttt{rm -r <cartella>}: rimuove una cartella.
\end{itemize}

\pagebreak
\section{Configurazione}
Prima di iniziare a utilizzare Git, è necessario configurare il proprio
nome e indirizzo email. Questi dati verranno utilizzati per identificare
i commit.
\begin{itemize}
    \item \texttt{git config --global user.name "Nome Cognome"}
    \item \texttt{git config --global user.email "Nostra Email"}
    \item \texttt{git config --list}: mostra la configurazione attuale.
    \item \texttt{git config --global --unset user.email}: rimuove un'impostazione.
    \item \texttt{git config --global --edit}: modifica il file di configurazione.
\end{itemize}

\section{Comandi base}
\begin{itemize}
    \item \texttt{git help}: mostra l'elenco dei comandi.
    \item \texttt{git init}: inizializza una nuova repository.
    \item \texttt{git status}: mostra lo stato della working directory.
    \item \texttt{git add <file>}: aggiunge un file alla staging area.
    \item \texttt{git commit -m "Messaggio"}: salva le modifiche.
    \item \texttt{git log}: mostra la cronologia dei commit.
\end{itemize}

\subsection{git add}
Il comando \texttt{git add} permette di aggiungere i file alla staging area.
\begin{itemize}
    \item \texttt{git add .}: aggiunge tutti i file modificati.
    \item \texttt{git add -A}: aggiunge tutti i file, anche quelli eliminati.
    \item \texttt{git add -u}: aggiunge solo i file modificati o eliminati.
    \item \texttt{git add <file>}: aggiunge un file specifico.
    \item \texttt{git add <cartella>}: aggiunge tutti i file della cartella.
\end{itemize}

\begin{highlight}
    Aggiungere un file alla staging area non comporta aggiungere le sue 
    successive modifiche
\end{highlight}

Esempio:
\begin{verbatim}
git add file.txt
git commit -m "Aggiunto file.txt"
\end{verbatim}    

A questo punto eseguendo
\begin{verbatim}
git status
\end{verbatim}
il file \texttt{file.txt} non sarà più presente nella lista dei file modificati 
(\texttt{Changes not staged for commit}) ma sarà presente nella lista dei file tracciati (\texttt{Changes to be committed}).

Supponiamo ora di modificare il file \texttt{file.txt} e di eseguire nuovamente
\begin{verbatim}
git status
\end{verbatim}
il file \texttt{file.txt} sarà presente sia nella lista dei file modificati 
 che in quella dei file tracciati.

Per far si che le modifiche vengano aggiunte alla staging area è necessario
eseguire nuovamente il comando \texttt{git add file.txt}.

\subsection{git commit}
Il comando \texttt{git commit} permette di salvare le modifiche nella repository.
\begin{itemize}
    \item \texttt{git commit -m "Messaggio"}: salva le modifiche con un messaggio.
    \item \texttt{git commit -am "Messaggio"}: aggiunge e salva i file modificati.
    \item \texttt{git commit --amend}: modifica l'ultimo commit.
    \item \texttt{git commit -v}: mostra le modifiche nel messaggio di commit.
\end{itemize}

\pagebreak
\subsection{git log}
Il comando \texttt{git log} mostra la cronologia dei commit.

\begin{itemize}
    \item \texttt{git log --oneline}: mostra i commit in una riga.
    \item \texttt{git log --stat}: mostra le statistiche dei file modificati.
    \item \texttt{git log --graph --oneline --all}: mostra il grafo dei commit.
    \item \texttt{git log --author="Nome Cognome"}: mostra i commit di un autore.
    \item \texttt{git log --grep="Messaggio"}: mostra i commit con un messaggio.
\end{itemize}

\section{Verifiche e ripensamenti}
\subsection{git commit --amend}
Supponiamo di aver appena fatto un commit, ma ci rendiamo
conto di aver fatto un piccolo errore da qualche parte.
Non vogliamo quindi fare un commit completamente nuovo, ma
modificare l'ultimo commit effettuato.

Per fare ciò, possiamo utilizzare il comando \texttt{git commit --amend}.
Questo comando aprirà un editor di testo in cui potremo modificare il messaggio del commit precedente.

In questo modo, invece di avere un commit in più, avremo semplicemente modificato il commit precedente.

Possiamo inoltre voler semplicemente modificare il messaggio 
del commit invece che il contenuto.

\begin{highlight}
    \texttt{git commit --amend}
\end{highlight}
Possiamo impostare il messaggio del commit direttamente da riga di comando
\begin{highlight}
    \texttt{git commit --amend -m "Nuovo messaggio"}
\end{highlight}

\subsection{git restore}
Mettiamo di aver aggiunto più file alla staging area di quelli che volevamo.
Possiamo rimuovere un file dalla staging area utilizzando il comando
\begin{highlight}
    \texttt{git restore --staged <file>}
\end{highlight}
Se vogliamo riportare uno o più file allo stato precedente, possiamo utilizzare
\begin{highlight}
    \texttt{git restore <file>}
\end{highlight}


\subsection{git diff}
Il comando \texttt{git diff} permette di visualizzare le differenze tra i file.
\begin{itemize}
    \item \texttt{git diff}: mostra le differenze tra la working directory e la staging area.
    \item \texttt{git diff --staged}: mostra le differenze tra la staging area e l'ultimo commit.
    \item \texttt{git diff <commit1> <commit2>}: mostra le differenze tra due commit.
    \item \texttt{git diff <file>}: mostra le differenze di un file.
\end{itemize}


\end{document}
 