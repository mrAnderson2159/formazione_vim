\documentclass[12pt]{article}
\usepackage[margin=1in]{geometry}
\usepackage{amsfonts,amsmath,amssymb}
\usepackage{graphicx}
\usepackage{tcolorbox}
\usepackage{hyperref}
\usepackage{fancyhdr}
\usepackage{float}
\usepackage{enumitem}
\usepackage{titlesec}
\usepackage{lmodern}
\usepackage{xcolor}

% Colori personalizzati
\definecolor{primary}{HTML}{2E86C1} % Blu moderno
\definecolor{secondary}{HTML}{117864} % Verde scuro
\definecolor{accent}{HTML}{D35400} % Arancione acceso
\definecolor{lightgray}{HTML}{F2F3F4} % Grigio chiaro per sfondi

% Impostazioni header e footer
\pagestyle{fancy}
\fancyhf{}
\fancyhead[L]{\textbf{\textcolor{primary}{Flutter}}}
\fancyhead[R]{\textcolor{secondary}{Valerio Molinari}}
\fancyfoot[C]{\thepage}

% Hyperlink
\hypersetup{
    colorlinks=true,
    linkcolor=primary,
    urlcolor=accent,
    citecolor=secondary
}

% Titoli stilizzati
\titleformat{\section}{\color{primary}\normalfont\Large\bfseries}{}{0em}{}
\titleformat{\subsection}{\color{secondary}\normalfont\large\bfseries}{}{0em}{}
\titleformat{\subsubsection}{\color{accent}\normalfont\normalsize\bfseries}{}{0em}{}

% Evidenziazione con box
\newtcolorbox{highlight}{colback=lightgray,colframe=primary!80!black,boxrule=0.5mm,arc=4mm,top=2mm,bottom=2mm,left=4mm,right=4mm}

% Spaziatura e interlinea
\setlength{\parskip}{1em}
\setlength{\parindent}{0pt}
\renewcommand{\baselinestretch}{1.5}

% Nuove definizioni
\newcommand{\important}[1]{\textcolor{accent}{\textbf{#1}}}

\begin{document}

% Copertina
\begin{titlepage}
\begin{center}
\vspace*{3cm}
\Huge\textcolor{primary}{\textbf{Formazione VIM}} \\[1cm]
\Large\textcolor{secondary}{Appunti di Flutter} \\[1cm]
\textcolor{accent}{Valerio Molinari}\\
\vfill
\today
\end{center}
\end{titlepage}

% Indice
\tableofcontents
\newpage

% Esempio di sezione
\section{Introduzione}
Flutter è un framework open-source sviluppato da Google per la creazione 
di applicazioni mobile per Android e iOS. 
Il linguaggio di programmazione utilizzato è Dart, 
anch'esso sviluppato da Google. 
Flutter permette di creare applicazioni native con un unico codice sorgente, 
garantendo prestazioni elevate e un'esperienza utente di qualità.

\begin{highlight}
Nota: è importante notare che è possibile scrivere il codice Dart su qualsiasi
macchina, ma il codice si può testare solo sulla macchina per cui è destinato.
\end{highlight}

\section{Installazione}
Segui le istruzioni sul sito, è complicato.

\section{Creazione di un progetto}
Per creare un nuovo progetto Flutter, apri il terminale e digita il seguente comando:
\begin{verbatim}
flutter create <nome_progetto>
\end{verbatim}
Questo comando crea una nuova cartella con il nome specificato, 
all'interno della quale vengono creati i file necessari per il progetto.

\section{Visual Studio Code \& Emulator} 
Una volta creato il progetto, apri Visual Studio Code e installa l'estensione
\textit{Flutter} per poter lavorare con il framework.
L'estensione permette di eseguire l'app direttamente su un emulatore,
senza dover passare per Android Studio o Xcode.

Per aprire l'emulatore su visual studio code,
vai sulla command palette, inizia a digitare il comando inserendo il simbolo 
$>$, quindi digita \textit{Flutter: Launch Emulator}.

\subsection{Eseguire l'applicazione}
Se non è già attiva, apri su VSC la barra di stato, che 
si attiva andando su \textit{Visualizza} > \textit{Aspetto} > \textit{Barra di stato},
oppure premendo \textit{Ctrl + Shift + P} e digitando \textit{Toggle Status Bar}.
Nella barra di stato, seleziona il dispositivo su cui vuoi 
eseguire l'applicazione.

A questo punto esistono tre modi per eseguire l'applicazione:
\begin{enumerate}
    \item Aprire il file \textit{main.dart} e cliccare su \textit{Run} sopra 
    la funzione \textit{main}.
    \item Aprire il terminale sulla cartella ed eseguire il comando 
    \textit{flutter run}.
    \item Cliccare sulla barra del menù \textit{Esegui} e selezionare
    \textit{Esegui senza debug}.
\end{enumerate}

\section{Funzioni}
In Dart, le funzioni prendono
parametri posizionali e non posizionali. Una funzione
viene dichiarata come:
\begin{verbatim}
tipo nomeFunzione({parametro1, parametro2}) { // questi sono parametri posizionali
  // corpo della funzione
}

tipo nomeFunzione2({parametro1, parametro2}) { // questi sono parametri non posizionali
  // corpo della funzione
}
\end{verbatim}

Queste vengono richiamate come segue:
\begin{verbatim} 
nomeFunzione(valore1, valore2);
nomeFunzione2(parametro1: valore1, parametro2: valore2);
\end{verbatim}


\section{main.dart}
Il file \textit{main.dart} è il file principale del progetto 
Flutter, in questa sezione vedremo come è strutturato.

\subsection{Struttura}
La funzione fondamentale di main.dart è la funzione 
\textit{runApp()}, che permette di eseguire l'applicazione.

Essa a sua volta deve essere chiamata all'interno di una funzione
\textit{main()}, che è la funzione principale del programma.

La funzione \textit{runApp()} non è \textit{nativa} di Dart,
pertanto è necessario importare il pacchetto \textit{flutter/material.dart}.

Un file dart ha quindi sempre una struttura del genere:
\begin{verbatim}
import 'package:flutter/material.dart';

void main() {
  runApp(...);
}
\end{verbatim}

\subsection{Widget}
In Flutter, tutto è un widget. Un widget è un elemento grafico
che può essere un bottone, un'immagine, un testo, ecc.

La funzione \textit{runApp()} prende in input un widget, che sarà
il widget principale dell'applicazione.

Fortunatamente Flutter è ben documentato e mette a disposizione
una vasta gamma di widget predefiniti, che possono essere combinati
per creare interfacce complesse. Questi widget si possono
trovare nella documentazione ufficiale di Flutter consultabile
all'indirizzo \url{https://docs.flutter.dev/ui/widgets}.

Tuttavia la funzione \textit{runApp()} non accetta un widget qualsiasi,
ma un widget di tipo \textit{Widget}, nello specifico un \textit{\textbf{MaterialApp}}.

\subsection{MaterialApp}
\textit{MaterialApp} è un widget che implementa il design di Material Design,
il design system di Google. Questo widget è il widget principale. Questa porzione di codice genera un titolo "Hello World" in cima ad un'applicazione vuota.
\begin{verbatim}
import 'package:flutter/material.dart';

void main() {
  runApp(MaterialApp(home: Text('Hello World!')));
}
\end{verbatim}

\subsection{const}
In Dart, la keyword \textit{const} viene utilizzata per creare
oggetti immutabili. Questo significa che una volta creato un oggetto con \textit{const}, non sarà possibile modificarlo.

Inoltre il risultato in memoria sarà più veloce, poiché l'oggetto
viene creato una sola volta e poi riferito.

\begin{verbatim}
  import 'package:flutter/material.dart';
  
  void main() {
    runApp(const juuMaterialApp(home: Text('Hello World!')));
  }
\end{verbatim}

\section{Widget}
Per creare un widget personalizzato, è necessario creare una nuova classe
che estende la classe \textit{StatelessWidget} o \textit{StatefulWidget}.

\subsection{StatelessWidget}
Un \textit{StatelessWidget} è un widget che non può cambiare il suo stato
durante l'esecuzione del programma. Questo significa che una volta creato
un \textit{StatelessWidget}, non sarà possibile modificarlo.

Un \textit{StatelessWidget} deve implementare il metodo \textit{build()} che
restituisce un widget. Poiché si tratta di 
una sovrascrittura, è necessario utilizzare l'annotazione \textit{@override}. Infine 
il metodo \textit{build()} deve prendere in input un oggetto di tipo \textit{BuildContext}.
\begin{verbatim}
class MyWidget extends StatelessWidget {
  @override
  Widget build(BuildContext context) {
    return Text('Hello World!');
  }
}
\end{verbatim}


\section{Modularizzazione}
Come per la maggior parte dei linguaggi di programmazione,
anche in Dart è possibile suddividere il codice in più file.
Per fare ciò, è necessario creare un nuovo file con estensione \textit{.dart}
e importare il file dove necessario.

Se supponiamo di avere una porzione di codice nel file
\textit{main.dart} che vogliamo spostare in un file separato,
creiamo un nuovo file \textit{mywidget.dart} e spostiamo il codice
al suo interno.

Per importarlo, dobbiamo usare la sintassi
\begin{highlight}
\begin{verbatim}
import 'package:<nome_progetto>/mywidget.dart';
\end{verbatim}
\end{highlight}


\section{Variabili}
Le variabili possono essere dichiarate al di fuori di una classe o al 
suo interno.
In Dart, le variabili possono essere di tipo \textit{var}, \textit{dynamic} o \textit{final}.

\subsection{var}
La keyword \textit{var} è utilizzata per dichiarare ed 
eventualmente definire una variabile il cui tipo
viene inferito dal compilatore. Questo significa che non è necessario specificare
il tipo della variabile, poiché il compilatore lo dedurrà in base al valore assegnato.

Inoltre una variabile di tipo \textit{var} può cambiare tipo o valore 
durante l'esecuzione del programma. 
Questo comporta che quando si cerca di utilizzare una variabile 
\textit{var} all'interno di un \texttt{const} Widget, si ottiene 
un errore, poiché il valore della variabile potrebbe cambiare e 
questo rende \textit{insicuro} il widget.

\subsection{dynamic}
Possiamo invece voler semplicemente dichiarare una variabile 
senza definirla. In questo caso possiamo sempre utilizzare \textit{var}
e il compilatore la dichiarerà come \textit{dynamic}.
Tuttavia, è possibile dichiarare una variabile di tipo \textit{dynamic}
esplicitamente.

\subsection{final}
Una variabile di tipo \textit{final} è una variabile costante,
il cui valore non può essere modificato una volta assegnato.
Questo significa che una variabile di tipo \textit{final} deve essere
inizializzata al momento della dichiarazione o nel costruttore.

\subsection{const}
Molto simile a \textit{final}, la keyword \textit{const} viene utilizzata
per creare variabili costanti. Tuttavia, a differenza di \textit{final},
una variabile di tipo \textit{const} deve essere inizializzata al momento
della dichiarazione, viene fissata dal compilatore ed 
è mantenuta in memoria per tutta la durata del programma.


\subsection{typed}
Una variabile può essere dichiarata con un tipo specifico,
come \textit{int}, \textit{double}, \textit{String}, ecc.
\begin{highlight}
\begin{verbatim}
int numero = 5;
double numeroDecimale = 5.5;
Alignment allineamento = Alignment.center;
\end{verbatim}
\end{highlight}
Se si dichiara una variabile con un tipo specifico, non sarà possibile
assegnare un valore di un tipo diverso. Se poi vogliamo dichiararne
il tipo senza assegnare un valore, possiamo utilizzare la sintassi
\begin{highlight}
\begin{verbatim}
Alignment? allineamento;
\end{verbatim}
\end{highlight}
In questo modo la variabile \textit{allineamento} sarà di tipo \textit{Alignment}
ma non avrà un valore assegnato.


\section{Parametri nei widget}
Un widget può, come qualsiasi classe, accettare parametri.
Dato un costruttore, possiamo definire una variabile
con un tipo e passarla al costruttore, vediamo 
un esempio mutuato dal Java.

\begin{highlight}
\begin{verbatim}
  // Java
  public class Persona {
    private String nome;
    private int eta;

    public Persona(String nome, int eta) {
      this.nome = nome;
      this.eta = eta;
    }
  }

  // Dart
  class Persona extends StatelessWidget {
    Persona(this.nome, this.eta, {super.key});

    String nome;
    int eta;
  }
\end{verbatim}
\end{highlight}


\section{Costruttori}
Il costruttore di default si chiama \textit{MyWidget()}
e si definisce come abbiamo gia visto: 
\begin{highlight}
\begin{verbatim}
class MyWidget extends StatelessWidget {
  MyWidget(posParam1, posParam2, {super.key, namedParam1, namedParam2});
  ...
}
\end{verbatim}
\end{highlight}

Tuttavia è possibile definire un costruttore personalizzato
che può avere parametri obbligatori o opzionali.

\begin{highlight}
\begin{verbatim}
class MyWidget extends StatelessWidget {
  MyWidget(posParam1, posParam2, {super.key, namedParam1, namedParam2});

  MyWidget.custom_name(posParam1 {super.key, namedParam1}) : 
    posParam2 = posParam1 + 1, namedParam2 = someValue;
  ...
}
\end{verbatim}
\end{highlight}

\section{Assets}
Per aggiungere immagini o altri file al progetto, è necessario
aggiungere il percorso del file al file \textit{pubspec.yaml}.
\begin{highlight}
\begin{verbatim}
flutter:
  assets:
    - assets/immagine.jpg
\end{verbatim}
\end{highlight}
















\end{document}
