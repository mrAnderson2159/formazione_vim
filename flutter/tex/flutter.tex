\documentclass[12pt]{article}
\usepackage[margin=1in]{geometry}
\usepackage{amsfonts,amsmath,amssymb}
\usepackage{graphicx}
\usepackage{tcolorbox}
\usepackage{hyperref}
\usepackage{fancyhdr}
\usepackage{float}
\usepackage{enumitem}
\usepackage{titlesec}
\usepackage{lmodern}
\usepackage{xcolor}

% Colori personalizzati
\definecolor{primary}{HTML}{2E86C1} % Blu moderno
\definecolor{secondary}{HTML}{117864} % Verde scuro
\definecolor{accent}{HTML}{D35400} % Arancione acceso
\definecolor{lightgray}{HTML}{F2F3F4} % Grigio chiaro per sfondi

% Impostazioni header e footer
\pagestyle{fancy}
\fancyhf{}
\fancyhead[L]{\textbf{\textcolor{primary}{Flutter}}}
\fancyhead[R]{\textcolor{secondary}{Valerio Molinari}}
\fancyfoot[C]{\thepage}

% Hyperlink
\hypersetup{
    colorlinks=true,
    linkcolor=primary,
    urlcolor=accent,
    citecolor=secondary
}

% Titoli stilizzati
\titleformat{\section}{\color{primary}\normalfont\Large\bfseries}{}{0em}{}
\titleformat{\subsection}{\color{secondary}\normalfont\large\bfseries}{}{0em}{}
\titleformat{\subsubsection}{\color{accent}\normalfont\normalsize\bfseries}{}{0em}{}

% Evidenziazione con box
\newtcolorbox{highlight}{colback=lightgray,colframe=primary!80!black,boxrule=0.5mm,arc=4mm,top=2mm,bottom=2mm,left=4mm,right=4mm}

% Spaziatura e interlinea
\setlength{\parskip}{1em}
\setlength{\parindent}{0pt}
\renewcommand{\baselinestretch}{1.5}

% Nuove definizioni
\newcommand{\important}[1]{\textcolor{accent}{\textbf{#1}}}

\begin{document}

% Copertina
\begin{titlepage}
\begin{center}
\vspace*{3cm}
\Huge\textcolor{primary}{\textbf{Formazione VIM}} \\[1cm]
\Large\textcolor{secondary}{Appunti di Flutter} \\[1cm]
\textcolor{accent}{Valerio Molinari}\\
\vfill
\today
\end{center}
\end{titlepage}

% Indice
\tableofcontents
\newpage

% Esempio di sezione
\section{Introduzione}
Flutter è un framework open-source sviluppato da Google per la creazione 
di applicazioni mobile per Android e iOS. 
Il linguaggio di programmazione utilizzato è Dart, 
anch'esso sviluppato da Google. 
Flutter permette di creare applicazioni native con un unico codice sorgente, 
garantendo prestazioni elevate e un'esperienza utente di qualità.

\begin{highlight}
Nota: è importante notare che è possibile scrivere il codice Dart su qualsiasi
macchina, ma il codice si può testare solo sulla macchina per cui è destinato.
\end{highlight}

\section{Installazione}
Segui le istruzioni sul sito, è complicato.

\section{Creazione di un progetto}
Per creare un nuovo progetto Flutter, apri il terminale e digita il seguente comando:
\begin{verbatim}
flutter create <nome_progetto>
\end{verbatim}
Questo comando crea una nuova cartella con il nome specificato, 
all'interno della quale vengono creati i file necessari per il progetto.

\section{Visual Studio Code \& Emulator} 
Una volta creato il progetto, apri Visual Studio Code e installa l'estensione
\textit{Flutter} per poter lavorare con il framework.
L'estensione permette di eseguire l'app direttamente su un emulatore,
senza dover passare per Android Studio o Xcode.

Per aprire l'emulatore su visual studio code,
vai sulla command palette, inizia a digitare il comando inserendo il simbolo 
$>$, quindi digita \textit{Flutter: Launch Emulator}.

\subsection{Eseguire l'applicazione}
Se non è già attiva, apri su VSC la barra di stato, che 
si attiva andando su \textit{Visualizza} > \textit{Aspetto} > \textit{Barra di stato},
oppure premendo \textit{Ctrl + Shift + P} e digitando \textit{Toggle Status Bar}.
Nella barra di stato, seleziona il dispositivo su cui vuoi 
eseguire l'applicazione.

A questo punto esistono tre modi per eseguire l'applicazione:
\begin{enumerate}
    \item Aprire il file \textit{main.dart} e cliccare su \textit{Run} sopra 
    la funzione \textit{main}.
    \item Aprire il terminale sulla cartella ed eseguire il comando 
    \textit{flutter run}.
    \item Cliccare sulla barra del menù \textit{Esegui} e selezionare
    \textit{Esegui senza debug}.
\end{enumerate}

\section{Funzioni}
In Dart, le funzioni prendono
parametri posizionali e non posizionali. Una funzione
viene dichiarata come:
\begin{verbatim}
tipo nomeFunzione({parametro1, parametro2}) { // questi sono parametri posizionali
  // corpo della funzione
}

tipo nomeFunzione2({parametro1, parametro2}) { // questi sono parametri non posizionali
  // corpo della funzione
}
\end{verbatim}

Queste vengono richiamate come segue:
\begin{verbatim} 
nomeFunzione(valore1, valore2);
nomeFunzione2(parametro1: valore1, parametro2: valore2);
\end{verbatim}


\section{main.dart}
Il file \textit{main.dart} è il file principale del progetto 
Flutter, in questa sezione vedremo come è strutturato.

\subsection{Struttura}
La funzione fondamentale di main.dart è la funzione 
\textit{runApp()}, che permette di eseguire l'applicazione.

Essa a sua volta deve essere chiamata all'interno di una funzione
\textit{main()}, che è la funzione principale del programma.

La funzione \textit{runApp()} non è \textit{nativa} di Dart,
pertanto è necessario importare il pacchetto \textit{flutter/material.dart}.

Un file dart ha quindi sempre una struttura del genere:
\begin{verbatim}
import 'package:flutter/material.dart';

void main() {
  runApp(...);
}
\end{verbatim}

\subsection{Widget}
In Flutter, tutto è un widget. Un widget è un elemento grafico
che può essere un bottone, un'immagine, un testo, ecc.

La funzione \textit{runApp()} prende in input un widget, che sarà
il widget principale dell'applicazione.

Fortunatamente Flutter è ben documentato e mette a disposizione
una vasta gamma di widget predefiniti, che possono essere combinati
per creare interfacce complesse. Questi widget si possono
trovare nella documentazione ufficiale di Flutter consultabile
all'indirizzo \url{https://docs.flutter.dev/ui/widgets}.

Tuttavia la funzione \textit{runApp()} non accetta un widget qualsiasi,
ma un widget di tipo \textit{Widget}, nello specifico un \textit{\textbf{MaterialApp}}.

\subsection{MaterialApp}
\textit{MaterialApp} è un widget che implementa il design di Material Design,
il design system di Google. Questo widget è il widget principale. Questa porzione di codice genera un titolo "Hello World" in cima ad un'applicazione vuota.
\begin{verbatim}
import 'package:flutter/material.dart';

void main() {
  runApp(MaterialApp(home: Text('Hello World!')));
}
\end{verbatim}

\subsection{const}
In Dart, la keyword \textit{const} viene utilizzata per creare
oggetti immutabili. Questo significa che una volta creato un oggetto con \textit{const}, non sarà possibile modificarlo.

Inoltre il risultato in memoria sarà più veloce, poiché l'oggetto
viene creato una sola volta e poi riferito.

\begin{verbatim}
  import 'package:flutter/material.dart';
  
  void main() {
    runApp(const juuMaterialApp(home: Text('Hello World!')));
  }
  \end{verbatim}












\end{document}
