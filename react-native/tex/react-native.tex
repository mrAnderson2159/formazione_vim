\documentclass[12pt]{article}
\usepackage[margin=1in]{geometry}
\usepackage{amsfonts,amsmath,amssymb}
\usepackage{graphicx}
\usepackage{tcolorbox}
\usepackage{hyperref}
\usepackage{fancyhdr}
\usepackage{float}
\usepackage{enumitem}
\usepackage{titlesec}
\usepackage{lmodern}
\usepackage{xcolor}

% Colori personalizzati
\definecolor{primary}{HTML}{2E86C1} % Blu moderno
\definecolor{secondary}{HTML}{117864} % Verde scuro
\definecolor{accent}{HTML}{D35400} % Arancione acceso
\definecolor{lightgray}{HTML}{F2F3F4} % Grigio chiaro per sfondi

% Impostazioni header e footer
\pagestyle{fancy}
\fancyhf{}
\fancyhead[L]{\textbf{\textcolor{primary}{React-Native}}}
\fancyhead[R]{\textcolor{secondary}{Valerio Molinari}}
\fancyfoot[C]{\thepage}

% Hyperlink
\hypersetup{
    colorlinks=true,
    linkcolor=primary,
    urlcolor=accent,
    citecolor=secondary
}

% Titoli stilizzati
\titleformat{\section}{\color{primary}\normalfont\Large\bfseries}{}{0em}{}
\titleformat{\subsection}{\color{secondary}\normalfont\large\bfseries}{}{0em}{}
\titleformat{\subsubsection}{\color{accent}\normalfont\normalsize\bfseries}{}{0em}{}

% Evidenziazione con box
\newtcolorbox{highlight}{colback=lightgray,colframe=primary!80!black,boxrule=0.5mm,arc=4mm,top=2mm,bottom=2mm,left=4mm,right=4mm}

% Spaziatura e interlinea
\setlength{\parskip}{1em}
\setlength{\parindent}{0pt}
\renewcommand{\baselinestretch}{1.5}

% Nuove definizioni
\newcommand{\important}[1]{\textcolor{accent}{\textbf{#1}}}

\begin{document}

% Copertina
\begin{titlepage}
\begin{center}
\vspace*{3cm}
\Huge\textcolor{primary}{\textbf{Formazione VIM}} \\[1cm]
\Large\textcolor{secondary}{Appunti di React-Native} \\[1cm]
\textcolor{accent}{Valerio Molinari}\\
\vfill
\today
\end{center}
\end{titlepage}

% Indice
\tableofcontents
\newpage

% Esempio di sezione
\section{Introduzione}
React-Native è un framework per lo sviluppo di applicazioni mobile, sviluppato da Facebook. 
Questo framework permette di creare applicazioni mobile per Android e iOS utilizzando JavaScript e React.

\subsection{Differenze tra React e React-Native}
Già dal principio possiamo notare alcune differenze tra i due linguaggi, soprattutto
per quanto riguarda i componenti. Infatti, mentre React utilizza componenti HTML, React-Native
utilizza componenti nativi per Android e iOS.

Laddove ad esempio in React si utilizza un tag \texttt{<div>}, in React-Native si utilizza un
\texttt{<View>}. Inoltre, mentre in React si utilizza un tag \texttt{<input>}, in React-Native si
utilizza un \texttt{<TextInput>} e così via.

Tutti i componenti di React-native possono essere trovati
e cosultati alla pagina \\\url{https://reactnative.dev/docs/components-and-apis}.

\subsection{Expo}
Expo è un framework che permette di sviluppare applicazioni React-Native senza la necessità di
installare Android Studio o XCode. Expo permette di testare le applicazioni dire
sul proprio smartphone, senza la necessità di emulatori.

Per iniziare a sviluppare con Expo, è sufficiente lanciare il comando
\begin{highlight}
\begin{verbatim}
npx create-expo-app@latest --template blank
\end{verbatim}
\end{highlight}
Usiamo \texttt{--template blank} per creare un progetto vuoto.

\end{document}
