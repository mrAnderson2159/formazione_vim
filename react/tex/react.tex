\documentclass[12pt]{article}
\usepackage[margin=1in]{geometry}
\usepackage{amsfonts,amsmath,amssymb}
\usepackage{graphicx}
\usepackage{tcolorbox}
\usepackage{hyperref}
\usepackage{fancyhdr}
\usepackage{float}
\usepackage{enumitem}
\usepackage{titlesec}
\usepackage{lmodern}
\usepackage{xcolor}

% Colori personalizzati
\definecolor{primary}{HTML}{2E86C1} % Blu moderno
\definecolor{secondary}{HTML}{117864} % Verde scuro
\definecolor{accent}{HTML}{D35400} % Arancione acceso
\definecolor{lightgray}{HTML}{F2F3F4} % Grigio chiaro per sfondi

% Impostazioni header e footer
\pagestyle{fancy}
\fancyhf{}
\fancyhead[L]{\textbf{\textcolor{primary}{React}}}
\fancyhead[R]{\textcolor{secondary}{Valerio Molinari}}
\fancyfoot[C]{\thepage}

% Hyperlink
\hypersetup{
    colorlinks=true,
    linkcolor=primary,
    urlcolor=accent,
    citecolor=secondary
}

% Titoli stilizzati
\titleformat{\section}{\color{primary}\normalfont\Large\bfseries}{}{0em}{}
\titleformat{\subsection}{\color{secondary}\normalfont\large\bfseries}{}{0em}{}
\titleformat{\subsubsection}{\color{accent}\normalfont\normalsize\bfseries}{}{0em}{}

% Evidenziazione con box
\newtcolorbox{highlight}{colback=lightgray,colframe=primary!80!black,boxrule=0.5mm,arc=4mm,top=2mm,bottom=2mm,left=4mm,right=4mm}

% Spaziatura e interlinea
\setlength{\parskip}{1em}
\setlength{\parindent}{0pt}
\renewcommand{\baselinestretch}{1.5}

% Nuove definizioni
\newcommand{\important}[1]{\textcolor{accent}{\textbf{#1}}}

\begin{document}

% Copertina
\begin{titlepage}
\begin{center}
\vspace*{3cm}
\Huge\textcolor{primary}{\textbf{Formazione VIM}} \\[1cm]
\Large\textcolor{secondary}{Appunti di React} \\[1cm]
\textcolor{accent}{Valerio Molinari}\\
\vfill
\today
\end{center}
\end{titlepage}

% Indice
\tableofcontents
\newpage

\section{Introduzione}
Si basa sull'utilizzo di file .jsx che sta per JavaScript XML. 
Questi file sono molto simili ai file HTML, 
ma con la differenza che possono contenere codice JavaScript. 
Questo codice viene poi trasformato in codice JavaScript puro 
tramite un compilatore chiamato Babel.

Questi file devono rispettare tre regole:
\begin{itemize}
    \item Devono contenere un solo elemento radice, ovvero una funzione
    che deve avere un nome che inizia per lettera maiuscola.
    \item La funzione deve restituire un valore, ovvero un elemento JSX.
    \item In fine devono avere una direttiva di esportazione 
    ({\tt export default <nome\_funzione>})
\end{itemize}

\section{Componenti}
I componenti sono delle funzioni che restituiscono un elemento JSX.
\begin{verbatim}
// src/compontents/Header.jsx
function Header() {
    function randomWorld() {
        const words = ["Fundamental", "Core", "Crucial"];

        return words[Math.round(Math.random() * (words.length - 1))];
    }

    const description = randomWorld();

    
    return (
        <header>
            <img src="src/assets/react-core-concepts.png" alt="Stylized atom" />
            <h1>React Essentials</h1>
            <p>
                {description} React concepts you will need for almost any app
                you are going to build! Let's get started!
            </p>
        </header>
    );
}

export default Header;

// src/App.jsx
import Header from "./components/Header";

function App() {
    return (
        <div>
            <Header />
            <main>
                <h2>Time to get started!</h2>
            </main>
        </div>
    );
}

export default App;
\end{verbatim}

\subsection{Importazioni}
Il comando {\tt import} permette di importare un modulo o un file,
ma permette anche di importare {\it media} come immagini, video, ecc.

Ad esempio possiamo correggere la riga del file {\tt Header.jsx}:
\begin{verbatim}
<img src="src/assets/react-core-concepts.png" alt="Stylized atom" />
\end{verbatim}
in:
\begin{verbatim}
    import logoImage from "../assets/react-core-concepts.png";

    ...
    
        return (
            <header>
                <img src={logoImage} alt="Stylized atom" />
                ...
            </header>
        );
    ...
    
    export default Header;
\end{verbatim}
Questo è importante perché in fase di deploy, React non sa dove trovare i file
multimediali, quindi è necessario importarli.

\subsubsection{Esportazioni nominali e default}
Generalmente le componenti vengono esportate come {\tt default},
per questo quando un file importa una componente, importa il nome
della componente, ma, sebbene sia una pratica sconsigliatissima,
è possibile importare una componente con un nome diverso da quello
della componente stessa.

Invece in un file so possono esportare più variabili {\it nominali},
tuttavia quando vengono importate, devono essere importate con lo stesso
nome con cui sono state esportate e {\bf devono essere racchiuse tra parentesi graffe}.




\section{Forward Props}
Supponiamo di avere una componente {\tt Card} 
che restituisce degli elementi HTML. Supponiamo di voler
applicare ad uno (o più) di questi elementi una classe CSS
che viene passata come parametro alla componente stessa e magari
un ID.

Nel provare ad applicare queste proprietà alla componente
{\tt Card} ci si accorge che queste non vengono automaticamente
passate agli elementi desiderati, e React si limiterà 
ad ignorarle nel DOM.

\begin{verbatim}
// src/App.jsx
import Card from "./components/Card";

export default function App() {
    return (
        <div>
            <Card className="card" id="card-1">
                <h2>Card Title</h2>
                <p>Card description</p>
            </Card>
        </div>
    );
}

// src/components/Card.jsx   <---- Errore
export default function Card({children}) {
    return (
        <div className={className} id={id}>
            {children}
        </div>
    );
}
\end{verbatim}
Un'alternativa sarebbe impostare esplicitamente questi parametri
nella signature della funzione, ma questo non è una buona pratica
perché rende la componente meno flessibile:

\begin{verbatim}
// src/components/Card.jsx

export default function Card({children, className, id}) {
    return (
        <div className={className} id={id}>
            {children}
        </div>
    );
}
\end{verbatim}

La soluzione consiste nell'utilizzare la proprietà del {\tt rest operator}
nativo di JavaScript, che permette di passare tutte le proprietà
che non sono state esplicitamente dichiarate nella signature della funzione
ad un oggetto.

\begin{verbatim}
// src/components/Card.jsx

export default function Card({children, ...props}) {
    return (
        <div id={props.id} className={props.className}>
            {children}
        </div>
    );
}
\end{verbatim}

O, se vogliamo passare tutte queste proprietà ad un singolo
elemento, che magari è proprio il contenitore della componente,
possiamo passare direttamente il {\tt rest operator} all'elemento
stesso.

\begin{verbatim}
// src/components/Card.jsx

export default function Card({children, ...props}) {
    return (
        <div {...props}>
            {children}
        </div>
    );
}
\end{verbatim}

\section{Slots}
I {\it slots} sono delle aree vuote all'interno di una componente
che possono essere riempite con altri elementi. 
Questo permette di creare componenti più flessibili e riutilizzabili.

Osserviamo questo codice dalla prima applicazione fatta in React:
\begin{verbatim}
// src/components/Example/Example.jsx
import { useState, useEffect } from "react";
import { EXAMPLES } from "../../data";
import TabButton from "../UI elements/TabButton";
import "./Example.css";

export default function Example() {
    const [activeTab, setActiveTab] = useState("");
    const [exemple, setExemple] = useState({});

    useEffect(() => {
        setExemple(EXAMPLES[activeTab.toLowerCase()]);
    }, [activeTab]);

    return (
        <>
            <menu>
                {Object.keys(EXAMPLES).map((example) => (
                    <TabButton
                        isSelected={activeTab === example}
                        clickHandler={setActiveTab}
                        key={example}
                    >
                        {example}
                    </TabButton>
                ))}
            </menu>
            <div id="tab-content">
                {!activeTab ? (
                    <p>Select an example to see its details.</p>
                ) : null}
                {exemple && (
                    <>
                        <h3>{exemple.title}</h3>
                        <p>{exemple.description}</p>
                        <pre>
                            <code>{exemple.code}</code>
                        </pre>
                    </>
                )}
            </div>
        </>
    );
}
\end{verbatim}

Possiamo notare che di base questo file contiene un \texttt{<menu>} 
che a sua volta contiene dei \texttt{<TabButton>}, e un \texttt{<div>}
che contiene il contenuto della tab selezionata.

Ha senso pensare che in un'applicazione più complessa, questa struttura
di menu e tab possa essere riutilizzata in altre parti dell'applicazione.
Per questo motivo, possiamo trasformare questa struttura in una componente
che accetta il contenuto come un {\it children}, mentre i pulsanti
possono essere passati come {\it slots}.

\pagebreak
A questo scopo creeremo la componente \texttt{Tabs}:
\begin{verbatim}
// src/components/UI elements/Tabs.jsx
export default function Tabs({ children, buttons }) {
    return (
        <>
            <menu>{buttons}</menu>
            {children}
        </>
    );
}

// src/components/Example/Example.jsx
import { useState, useEffect } from "react";
import { EXAMPLES } from "../../data";
import TabButton from "../UI elements/TabButton";
import Tabs from "../UI elements/Tabs";
import "./Example.css";

export default function Example() {
    const [activeTab, setActiveTab] = useState("");
    const [exemple, setExemple] = useState({});

    useEffect(() => {
        setExemple(EXAMPLES[activeTab.toLowerCase()]);
    }, [activeTab]);

    const tabsContent = (
        <div id="tab-content">
            {!activeTab ? <p>Select an example to see its details.</p> : null}
            {exemple && (
                <>
                    <h3>{exemple.title}</h3>
                    <p>{exemple.description}</p>
                    <pre>
                        <code>{exemple.code}</code>
                    </pre>
                </>
            )}
        </div>
    );

    const buttons = Object.keys(EXAMPLES).map((example) => (
        <TabButton
            isSelected={activeTab === example}
            clickHandler={setActiveTab}
            key={example}
        >
            {example}
        </TabButton>
    ));

    return (
        <>
            <Tabs buttons={buttons}>{tabsContent}</Tabs>
        </>
    );
}
\end{verbatim}

\pagebreak
Qui abbiamo semplicemente salvato la logica precedente in due variabili:
\texttt{tabsContent} e \texttt{buttons}, quindi abbiamo passato queste
variabili come {\it slot} e {\it children} alla componente \texttt{Tabs}.






















































\end{document}
