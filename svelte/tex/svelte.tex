\documentclass[12pt]{article}
\usepackage[margin=1in]{geometry}
\usepackage{amsfonts,amsmath,amssymb}
\usepackage{graphicx}
\usepackage{tcolorbox}
\usepackage{hyperref}
\usepackage{fancyhdr}
\usepackage{float}
\usepackage{enumitem}
\usepackage{titlesec}
\usepackage{lmodern}
\usepackage{xcolor}

% Colori personalizzati
\definecolor{primary}{HTML}{2E86C1} % Blu moderno
\definecolor{secondary}{HTML}{117864} % Verde scuro
\definecolor{accent}{HTML}{D35400} % Arancione acceso
\definecolor{lightgray}{HTML}{F2F3F4} % Grigio chiaro per sfondi

% Impostazioni header e footer
\pagestyle{fancy}
\fancyhf{}
\fancyhead[L]{\textbf{\textcolor{primary}{Svelte}}}
\fancyhead[R]{\textcolor{secondary}{Valerio Molinari}}
\fancyfoot[C]{\thepage}

% Hyperlink
\hypersetup{
    colorlinks=true,
    linkcolor=primary,
    urlcolor=accent,
    citecolor=secondary
}

% Titoli stilizzati
\titleformat{\section}{\color{primary}\normalfont\Large\bfseries}{}{0em}{}
\titleformat{\subsection}{\color{secondary}\normalfont\large\bfseries}{}{0em}{}
\titleformat{\subsubsection}{\color{accent}\normalfont\normalsize\bfseries}{}{0em}{}

% Evidenziazione con box
\newtcolorbox{highlight}{colback=lightgray,colframe=primary!80!black,boxrule=0.5mm,arc=4mm,top=2mm,bottom=2mm,left=4mm,right=4mm}

% Spaziatura e interlinea
\setlength{\parskip}{1em}
\setlength{\parindent}{0pt}
\renewcommand{\baselinestretch}{1.5}

% Nuove definizioni
\newcommand{\important}[1]{\textcolor{accent}{\textbf{#1}}}

\begin{document}

% Copertina
\begin{titlepage}
\begin{center}
\vspace*{3cm}
\Huge\textcolor{primary}{\textbf{Formazione VIM}} \\[1cm]
\Large\textcolor{secondary}{Appunti di Svelte} \\[1cm]
\textcolor{accent}{Valerio Molinari}\\
\vfill
\today
\end{center}
\end{titlepage}

% Indice
\tableofcontents
\newpage

% Esempio di sezione
\section{Introduzione}
Svelte è un framework per la creazione di applicazioni web. 
È un'applicazione web che si occupa di gestire la parte front-end 
di un'applicazione web.

% \begin{highlight}
% Questa è un'area evidenziata per definizioni importanti o esempi pratici.
% \end{highlight}

\subsection{Creare un progetto}
Il modo più semplice per creare un progetto Svelte è 
lanciare i seguenti comandi:
\begin{verbatim}
    npx sv create <my-app>
    cd <my-app>
    npm install
    npm run dev
\end{verbatim}
Questo ci permetterà di creare un progetto Svelte e di ottenere l'indirizzo
locale del server di sviluppo.

\section{Svelte file}
Un file Svelte è composto da tre parti principali:
\begin{itemize}
    \item \textbf{Script}: contiene il codice JavaScript che gestisce la logica dell'applicazione.
    \item \textbf{Style}: contiene il codice CSS che gestisce lo stile dell'applicazione.
    \item \textbf{Template}: contiene il codice HTML che definisce la struttura dell'applicazione.
\end{itemize}

\pagebreak
Esempio di file Svelte:
\begin{verbatim}
<script>
    let count = 0;
    function onclick() {
        count += 1;
    }
</script>

<style>
    button {
        color: white;
        background-color: blue;
    }
</style>

<button {onclick}>
    Clicked {count} {count === 1 ? 'time' : 'times'}
</button>
\end{verbatim}

Mentre lo script e lo stile si spiegano da soli, possiamo notare nel template 
l'utilizzo della shortcut {\tt \{onclick\}} che permette di associare la funzione
{\tt onclick} all'evento {\tt click} del bottone, senza dover scrivere onclick={onclick}.

\begin{highlight}
    Alla creazione di un nuovo progetto, svelte creerà il file src/routes/+page.svelte 
    che contiene il codice della pagina web all'indirizzo del server locale.
\end{highlight}

\subsection{\$state e \$derived}
Svelte mette a disposizione due variabili speciali:
\begin{itemize}
    \item \important{\$state}: contiene lo stato attuale dell'applicazione.
    \item \important{\$derived}: contiene le variabili derivate dallo stato attuale.
\end{itemize}

Supponiamo di avere il seguente codice:
\begin{verbatim}
<script lang='ts'>
    let number = 0;
    function onclick() {
        number++;
    }
</script>

<button {onclick}>
    Click me!
</button>

<h1>{number}</h1>
\end{verbatim}
e supponiamo di voler mostrare all'utente un messaggio personalizzato
in base al numero di click.

Un esempio classico potrebbe essere
\begin{verbatim}
<script lang='ts'>
    let number = 0;
    function onclick() {
        number++;
    }
</script>

<button {onclick}>
    Click me!
</button>

<h1>{number}</h1>
<p>{number === 0 ? 
"Hey! Why don't you try to click the button" :
`You clicked ${number} times`}</p>
\end{verbatim}

Ma una soluzione più compatta e più nello stile di \texttt{Svelte} consiste
nel registrare lo stato \texttt{number} come \texttt{\$state} e 
creare una variabile derivata \texttt{message} che si aggiorna automaticamente
in base allo stato di \texttt{number}.

\begin{verbatim}
<script lang='ts'>
    let number = $state(0);
    let userInformation = $derived(
        number === 0 ? "Hey! Why don't you try to click the button" :
                       `You clicked ${number} times`
    )
    function onclick() {
        number++;
    }
</script>

<button {onclick}>
    Click me!
</button>

<h1>{number}</h1>
<p>{userInformation}</p>
\end{verbatim}

O ancora possiamo creare una funzione apposita per elaborare il messaggio 
e chiamarla nella callback della derivata, utilizzando {\tt \$derived.by}.

\begin{verbatim}
<script lang='ts'>
    let number = $state(0);
    let userInformation = $derived.by(() => calculateUserInfo(number));
    function onclick() {
        number++;
    }
    function calculateUserInfo(number: number) {
        switch (number) {
            case 0:
                return "Hey! Why don't you try to click the button";
            case 1:
                return "You clicked exactly one time";
            default:
                return `You clicked ${number} times`;
        }
    }
</script>

<button {onclick}>
    Click me!
</button>

<h1>{number}</h1>
<p>{userInformation}</p>
\end{verbatim}

\subsection{Binding}
Svelte permette di associare una variabile ad un elemento HTML in modo che
ogni modifica della variabile si rifletta automaticamente sull'elemento.

Supponiamo di avere il seguente codice:
\begin{verbatim}
<script lang='ts'>
    let username = $state('');
</script>   

<input type='text' />
<h1>Hello {username}</h1>
\end{verbatim}

Per associare la variabile \texttt{username} all'input, 
possiamo utilizzare la direttiva \texttt{bind:value}.
In questo modo ogni modifica dell'input si rifletterà automaticamente
sulla variabile \texttt{username}.

\begin{verbatim}
<script lang='ts'>
    let username = $state('');
</script>

<input type='text' bind:value={username} />
<h1>Hello {username}</h1>
\end{verbatim}

\subsection{\$effect}
Svelte mette a disposizione la variabile \important{\$effect} che permette di
eseguire un effetto collaterale ogni volta che una variabile cambia.
Consideriamo di voler stampare a console il valore di \texttt{username} ogni volta
che questo cambia.

\begin{verbatim}
<script lang='ts'>
    let username = $state('');
    $effect(() => {
        console.log(username);
    });
</script>

<input type='text' bind:value={username} />
<h1>Hello {username}</h1>
\end{verbatim}
\begin{highlight}
    \$effect è da considerarsi come una \textit{ultima risorsa}
    in quanto "esula" dalla filosofia di Svelte
\end{highlight}

\subsection{\$inspect}
Svelte mette a disposizione un utile strumento di debug,
ovvero la variabile \important{\$inspect} che permette di visualizzare
lo stato attuale di una variabile.
\begin{verbatim}
<script lang='ts'>
    let username = $state('');
    $inspect(username);
</script>

<input type='text' bind:value={username} />
<h1>Hello {username}</h1>
\end{verbatim}
Questo codice ci permette di vedere qualsiasi modifica alla 
variabile \texttt{username} nella console del browser.

\section{Componenti}
Svelte permette di creare componenti riutilizzabili.
Un componente è un file Svelte che può essere importato in un altro file Svelte.

\subsection{Creare un componente}
Per creare un componente, basta creare un file Svelte con il codice del componente
e importarlo nel file principale.
\begin{verbatim}
// src/components/userInpit.svelte

<script lang='ts'>
    let username = $state('');
</script>

<input type='text' bind:value={username} />
<h1>Hello {username}</h1>


// src/routes/+page.svelte

<script lang='ts'>
    import UserInput from '../components/userInput.svelte';
</script>

<UserInput />
\end{verbatim}
Possiamo poi passare delle proprietà al componente, in modo da renderlo più flessibile.

Tali proprietà possono essere di due tipi, normali o children:
\begin{highlight}
    \begin{verbatim}
        <Component prop1={value1} prop2={value2}>
            <h1>Children</h1>
        </Component>
    \end{verbatim}
\end{highlight}

\subsection{Proprietà normali}
Supponiamo quindi di voler utilizzare la componente \texttt{UserInput}
in modo tale da passare un valore di default per l'input.

\begin{verbatim}
// src/components/userInpit.svelte

<script lang='ts'>
    interface UserInputProps {
        username: string;
    }

    let { username }: UserInputProps = $props();
</script>

<input type="text" bind:value={username} placeholder="Enter your name" />

<h1>{username}</h1>


// src/routes/+page.svelte
<script>
    import UserInput from "$lib/components/userInput.svelte";
</script>

<UserInput username={"mario"} />
\end{verbatim}

Qui abbiamo definito un'interfaccia \texttt{UserInputProps} che definisce
le proprietà che il componente accetta. 
Abbiamo poi utilizzato la funzione \texttt{\$props()} per ottenere le proprietà
passate al componente.

\subsection{Proprietà children}
Supponiamo poi di voler stampare un messaggio tramite la nostra 
componente \texttt{UserInput}.

Per farlo dobbiamo definire nell'interfaccia \texttt{children}
come uno \texttt{Snippet} e utilizzare la direttiva \texttt{@render}
per renderizzare il contenuto nella componente. Vedremo meglio
entrambi i concetti nei prossimi capitoli.

\begin{verbatim}
// src/components/userInpit.svelte

<script lang='ts'>
	import type { Snippet } from "svelte";

    interface UserInputProps {
        children: Snippet;
        username: string;
    }

    let { username, children }: UserInputProps = $props();
</script>

<input type="text" bind:value={username} placeholder="Enter your name" />

<h1>{username}</h1>

<p>
    {@render children()}
</p>


// src/routes/+page.svelte
<script>
    import UserInput from "$lib/components/userInput.svelte";
</script>

<UserInput username={"mario"}>
    <h2>Messaggio renderizzato dalla componente</h2>
</UserInput>
\end{verbatim}

\subsection{Snippet}
Esiste in svelte la possibilità di creare una sorta di componente
all'interno di uno stesso file piuttosto che in un file
separato. Questo si chiama \important{Snippet}.

Consiste in un frammento di html che può essere utilizzato
all'interno di un file svelte, dislocando la logica di script
e stile all'interno del file principale.

Se volessimo fare un parallelismo con un qualsiasi linguaggio 
OOP, se una componente fosse una classe, uno snippet sarebbe
una funzione.

\pagebreak
Vediamo nel dettaglio un esempio di snippet:
\begin{verbatim}
<script lang="ts">
    let userName = $state('Valerio');
</script>

{#snippet userInput(exempleString: string)}
    <h1>{exempleString}</h1>
    <h2>Your username</h2>
    <input type="text" bind:value={userName} placeholder="Enter your name" />
    <p>{userName}</p>
{/snippet}

{@render userInput('Example 1')}
\end{verbatim}
Analizziamo il codice:
\begin{itemize}
    \item Abbiamo definito lo script nel linguaggio TypeScript
    per definire il parametro exempleString come stringa.
    \item Abbiamo definito uno snippet chiamato \texttt{userInput} che accetta
    una stringa come parametro.
    \item Abbiamo usato la variabile \texttt{userName} 
    all'interno dello snippet poiché esso ha accesso a 
    tutte le variabili dello script.
    \item Abbiamo utilizzato la direttiva \texttt{@render} per renderizzare
    lo snippet \texttt{userInput} passando come parametro la stringa \texttt{'Example 1'}.
\end{itemize}

\subsection{Conditional rendering}
Supponiamo di avere un pulsante per modificare l'username.
Noi potremmo mettere sulla stessa pagina sia l'username
che un campo input per modificarlo, ma potremmo scegliere
quale dei due mostrare in base ad una variabile booleana.
\begin{verbatim}
    <script lang="ts">
    let userName = $state('Valerio');
    let isEditMode = $state(false);
</script>

{#snippet userInput(exempleString: string)}
    <div class="container">
        <h1>{exempleString}</h1>
        <h2>Your username</h2>

        {#if isEditMode}
            <div class="input_container">
                <input type="text" bind:value={userName} placeholder="Enter your name" />
            </div>
        {:else}
            <p>{userName}</p>
        {/if}
    </div>
{/snippet}

{@render userInput('Example 1')}

<button onclick={() => isEditMode = !isEditMode}>
    {isEditMode ? 'Save changes' : 'Edit fields'}
</button>

<style>
    .container {
        display: flex;
        flex-direction: column;
    }

    .container h1 {
        margin: 0 auto;
    }

    .input_container {
        margin-top: 1rem;
        margin-bottom: 1rem;
    }
</style>
\end{verbatim}
Analizziamo il codice:
\begin{itemize}
    \item Abbiamo definito una variabile \texttt{isEditMode} che ci permette
    di scegliere se mostrare l'input o l'username.
    \item Abbiamo utilizzato la direttiva \texttt{\#if} per mostrare l'input
    se \texttt{isEditMode} è \texttt{true} e l'username altrimenti.
    \item Abbiamo definito un bottone che cambia il valore di \texttt{isEditMode}
    ad ogni click.
\end{itemize}

\pagebreak
\subsection{Each}
Supponiamo di avere una lista di utenti e di voler mostrare
una lista di username.
\begin{verbatim}
    <script lang="ts">
    let userName = $state('Valerio');
    let isEditMode = $state(false);
    let peopleWaiting = $state(['Mario', 'Luigi', 'Peach', 'Toad']);
</script>

{#snippet userInput(exempleString: string)}
    <div class="container">
        <h1>{exempleString}</h1>
        <h2>Your username</h2>

        {#if isEditMode}
            <div class="input_container">
                <input type="text" bind:value={userName} placeholder="Enter your name" />
            </div>
        {:else}
            <p>{userName}</p>
        {/if}
    </div>
{/snippet}

{@render userInput('Example 1')}

<button onclick={() => isEditMode = !isEditMode}>
    {isEditMode ? 'Save changes' : 'Edit fields'}
</button>

<h2>People waiting</h2>
<ul>
    {#each peopleWaiting as person}
        <li>{person}</li>
    {/each}
</ul>

<button onclick={() => peopleWaiting.push(userName)}>I'm waiting too</button>
\end{verbatim}
Analizziamo il codice:
\begin{itemize}
    \item Abbiamo definito una variabile \texttt{peopleWaiting} che contiene
    una lista di stringhe.
    \item Abbiamo utilizzato la direttiva \texttt{\#each} per mostrare
    ogni elemento della lista \texttt{peopleWaiting}.
    \item Abbiamo definito un bottone che aggiunge l'username all'array
    \texttt{peopleWaiting} ad ogni click.
\end{itemize}



\pagebreak
\section{SvelteKit}
SvelteKit è un framework per la creazione di applicazioni web.
Si occupa di gestire la parte back-end
dell'applicazione.
\begin{highlight}
    SvelteKit viene installato automaticamente in un'applicazione
    Svelte
\end{highlight}
\subsection{Componenti Frontend}
Le componenti frontend in \textbf{SvelteKit} sono rappresentate
principalmente dai file che iniziano con il simbolo \texttt{+}
e che si trovano all'interno della cartella \texttt{src/routes}
o nelle sue sottodirectory.

\subsubsection{Page}
Alla creazione di una nuova app \textbf{SvelteKit}, nella cartella
\texttt{src/routes} viene generato il file \texttt{+page.svelte},
che rappresenta la pagina principale dell'applicazione.
Sebbene possa sembrare l'entry point dell'app, in realtà
il vero entry point è il file \texttt{src/app.html}, che
si occupa di caricare la pagina principale all'interno del
\texttt{body} dell'HTML.

La struttura di \texttt{+page.svelte} è ricorrente ed è
il meccanismo con cui \textbf{SvelteKit} gestisce il 
\textit{routing} dell’applicazione.

\begin{itemize}
    \item La pagina principale, accessibile all'indirizzo \texttt{/},
    è definita nel file \texttt{routes/+page.svelte}.
    \item Se vogliamo una pagina accessibile all’indirizzo \texttt{/blog},
    questa dovrà essere definita nel file \texttt{routes/blog/+page.svelte}.
\end{itemize}

La gerarchia delle cartelle in \texttt{src/routes} riflette
direttamente la struttura degli URL, facilitando la gestione
delle pagine.

\subsubsection{Layout}  
In \textbf{SvelteKit}, un \textit{layout} definisce una
struttura comune per un gruppo di pagine correlate.

Se creiamo un file \texttt{+layout.svelte} all'interno
della cartella \texttt{src/routes/blog}, tutte le pagine
contenute in \texttt{src/routes/blog} erediteranno
automaticamente quel layout.

Se una sottocartella di \texttt{src/routes/blog} definisce
un proprio layout, verrà applicata una gerarchia di layout
annidati, con ciascun livello che eredita quelli superiori.

\begin{highlight}
    È essenziale ricordarsi di renderizzare i \textit{children}!
\end{highlight}

Un aspetto cruciale dei layout è la necessità di renderizzare
il contenuto delle pagine figlie. Se omettiamo questo passaggio,
navigando in un ramo vedremo soltanto il layout, senza
il contenuto della pagina stessa.

Per risolvere questo problema, dobbiamo sempre includere il
seguente codice nel nostro layout:

\begin{verbatim}
<script lang='ts'>
    const data = $props();
</script>

...

{@render data.children()}

... 
\end{verbatim}

\subsubsection{Pagine dinamiche}
Se vogliamo creare un sistema di articoli raggiungibili
tramite URL dinamici, come \texttt{/blog/n}, \textbf{SvelteKit}
ci consente di farlo creando una cartella speciale:

\texttt{src/routes/blog/[articleid]}.

\begin{highlight}
    La cartella \texttt{[articleid]} è una cartella speciale
    che consente di creare pagine dinamiche. Deve essere
    racchiusa tra parentesi quadre.
\end{highlight}

\subsection{Pagine TypeScript}
\textbf{SvelteKit} consente di scrivere pagine direttamente in
\textbf{TypeScript}, permettendo l'esecuzione sia lato server
che lato client.

\subsection{Componenti Backend}
Le componenti backend in \textbf{SvelteKit} sono costituite
dai file che contengono la parola \texttt{server} nel nome.

\subsubsection{Page Server}
Un file come \texttt{+page.server.ts} viene eseguito lato server.
Può essere utilizzato per operazioni non eseguibili lato client,
come l'accesso al database o le chiamate API.

Se ci troviamo nella cartella \texttt{routes/blog/[articleid]}
e creiamo il file \texttt{+page.server.ts} accanto a
\texttt{+page.svelte}, possiamo definire una funzione asincrona
per caricare i dati lato server:

\begin{verbatim}
import type { PageServerLoad } from './$types';

export const load: PageServerLoad = async ({ params }) => {
    console.log(params);

    return {
        blogPost: 'This is the example blog post'
    };
};
\end{verbatim}

Quando il client effettua una richiesta \texttt{GET /blog/[articleid]},
il server eseguirà la funzione \texttt{load}, e nella console
vedremo un output simile a:

\begin{verbatim}
{ articleid: 'n' }
\end{verbatim}

Se poi nel file \texttt{routes/blog/[articleid]/+page.svelte}
aggiungiamo:

\begin{verbatim}
<script lang='ts'>
    const data = $props();
    $inspect(data);
</script>

<h1>Blog Article</h1>    
\end{verbatim}

Vedremo comparire sulla console del server e del browser
i dati ricevuti dal backend:

\begin{verbatim}
init { data: { blogPost: 'This is the example blog post' }, form: null }
\end{verbatim}

Ciò significa che \texttt{data.blogPost} è stato
correttamente fornito al frontend dal backend.

Tuttavia, nella maggior parte dei casi reali, i dati non
sono semplicemente statici, ma provengono da un database
oppure da un'API esterna. Per simulare un contesto più
realistico, possiamo immaginare di avere un array di
articoli già disponibili e di voler recuperare quello
corrispondente all'\texttt{articleid} ricevuto nella
richiesta.

Ecco un esempio pratico:
\begin{verbatim}
// src/routes/blog/[articleid]/+page.server.ts
import { error } from '@sveltejs/kit';
import type { PageServerLoad } from './$types';

export const load: PageServerLoad = async ({ params }) => {
	const fetchedArticles = [
		{
			id: '0',
			text: "This is the first article's text"
		},
		{
			id: '1',
			text: "This is the second article's text"
		}
	];

	const findArticle = fetchedArticles.find((article) => article.id === params.articleid);

	if (findArticle) {
		return {
			blogPost: findArticle.text
		};
	}

	throw error(404, 'Article not found');
};
\end{verbatim}
\pagebreak
\begin{verbatim}
// src/routes/blog/[articleid]/+page.svelte
<script lang='ts'>
    const {data} = $props();

    const {blogPost} = data;
</script>

<h1>Blog Article</h1>
<p>{blogPost}</p>
\end{verbatim}
In questo modo, il contenuto dell'articolo selezionato
verrà mostrato dinamicamente all'interno della pagina.


\section{API e CRUD}
Quando vogliamo implementare logica puramente di 
backend, come l'accesso al database o le chiamate API,
possiamo creare la directory \texttt{src/routes/api},
al cui interno creeremo una cartella per ogni endpoint
backend, quindi un file \texttt{+server.ts} per ogni
endpoint.

Un semplice esempio potrebbe essere un endpoint che
conferma l'acquisto di un ebook da parte di un cliente.
Questo si troverebbe all'indirizzo
\texttt{/api/purchase-confirmation} stipato quindi 
nel seguente file
\begin{verbatim}
// src/routes/api/purchase-confirmation/+server.ts
import { json } from '@sveltejs/kit';

export async function POST({ request }) {
	const bodyRequest = await request.json();

	console.log(bodyRequest);

	// Business logic -> send an email to the costumer
	//      and send them the bought ebook

	return json({ message: 'Email sent to the costumer, all good.' });
}
\end{verbatim}




















\pagebreak
\section{Note}
\subsection{Design}
Se serve ispirazione per il design, si consiglia di visitare
\href{https://www.tailwindcss.com}{Tailwind CSS}.

Se serve ispirazione peru una landing page, si consiglia di visitare
\href{https://dribbble.com}{Dribbble}.

\end{document}