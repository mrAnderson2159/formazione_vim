\documentclass[12pt]{article}
\usepackage[margin=1in]{geometry}
\usepackage{amsfonts,amsmath,amssymb}
\usepackage{graphicx}
\usepackage{tcolorbox}
\usepackage{hyperref}
\usepackage{fancyhdr}
\usepackage{float}
\usepackage{enumitem}
\usepackage{titlesec}
\usepackage{lmodern}
\usepackage{xcolor}

% Colori personalizzati
\definecolor{primary}{HTML}{2E86C1} % Blu moderno
\definecolor{secondary}{HTML}{117864} % Verde scuro
\definecolor{accent}{HTML}{D35400} % Arancione acceso
\definecolor{lightgray}{HTML}{F2F3F4} % Grigio chiaro per sfondi

% Impostazioni header e footer
\pagestyle{fancy}
\fancyhf{}
\fancyhead[L]{\textbf{\textcolor{primary}{Svelte}}}
\fancyhead[R]{\textcolor{secondary}{Valerio Molinari}}
\fancyfoot[C]{\thepage}

% Hyperlink
\hypersetup{
    colorlinks=true,
    linkcolor=primary,
    urlcolor=accent,
    citecolor=secondary
}

% Titoli stilizzati
\titleformat{\section}{\color{primary}\normalfont\Large\bfseries}{}{0em}{}
\titleformat{\subsection}{\color{secondary}\normalfont\large\bfseries}{}{0em}{}
\titleformat{\subsubsection}{\color{accent}\normalfont\normalsize\bfseries}{}{0em}{}

% Evidenziazione con box
\newtcolorbox{highlight}{colback=lightgray,colframe=primary!80!black,boxrule=0.5mm,arc=4mm,top=2mm,bottom=2mm,left=4mm,right=4mm}

% Spaziatura e interlinea
\setlength{\parskip}{1em}
\setlength{\parindent}{0pt}
\renewcommand{\baselinestretch}{1.5}

% Nuove definizioni
\newcommand{\important}[1]{\textcolor{accent}{\textbf{#1}}}

\begin{document}

% Copertina
\begin{titlepage}
\begin{center}
\vspace*{3cm}
\Huge\textcolor{primary}{\textbf{Formazione VIM}} \\[1cm]
\Large\textcolor{secondary}{Appunti di Svelte} \\[1cm]
\textcolor{accent}{Valerio Molinari}\\
\vfill
\today
\end{center}
\end{titlepage}

% Indice
\tableofcontents
\newpage

% Esempio di sezione
\section{Introduzione}
Svelte è un framework per la creazione di applicazioni web. 
È un'applicazione web che si occupa di gestire la parte front-end 
di un'applicazione web.

% \begin{highlight}
% Questa è un'area evidenziata per definizioni importanti o esempi pratici.
% \end{highlight}

\subsection{Creare un progetto}
Il modo più semplice per creare un progetto Svelte è 
lanciare i seguenti comandi:
\begin{verbatim}
    npx sv create <my-app>
    cd <my-app>
    npm install
    npm run dev
\end{verbatim}
Questo ci permetterà di creare un progetto Svelte e di ottenere l'indirizzo
locale del server di sviluppo.

\section{Svelte file}
Un file Svelte è composto da tre parti principali:
\begin{itemize}
    \item \textbf{Script}: contiene il codice JavaScript che gestisce la logica dell'applicazione.
    \item \textbf{Style}: contiene il codice CSS che gestisce lo stile dell'applicazione.
    \item \textbf{Template}: contiene il codice HTML che definisce la struttura dell'applicazione.
\end{itemize}

\pagebreak
Esempio di file Svelte:
\begin{verbatim}
<script>
    let count = 0;
    function onclick() {
        count += 1;
    }
</script>

<style>
    button {
        color: white;
        background-color: blue;
    }
</style>

<button {onclick}>
    Clicked {count} {count === 1 ? 'time' : 'times'}
</button>
\end{verbatim}

Mentre lo script e lo stile si spiegano da soli, possiamo notare nel template 
l'utilizzo della shortcut {\tt \{onclick\}} che permette di associare la funzione
{\tt onclick} all'evento {\tt click} del bottone, senza dover scrivere onclick={onclick}.

\begin{highlight}
    Alla creazione di un nuovo progetto, svelte creerà il file src/routes/+page.svelte 
    che contiene il codice della pagina web all'indirizzo del server locale.
\end{highlight}

\subsection{\$state e \$derived}
Svelte mette a disposizione due variabili speciali:
\begin{itemize}
    \item \important{\$state}: contiene lo stato attuale dell'applicazione.
    \item \important{\$derived}: contiene le variabili derivate dallo stato attuale.
\end{itemize}

Supponiamo di avere il seguente codice:
\begin{verbatim}
<script lang='ts'>
    let number = 0;
    function onclick() {
        number++;
    }
</script>

<button {onclick}>
    Click me!
</button>

<h1>{number}</h1>
\end{verbatim}
e supponiamo di voler mostrare all'utente un messaggio personalizzato
in base al numero di click.

Un esempio classico potrebbe essere
\begin{verbatim}
<script lang='ts'>
    let number = 0;
    function onclick() {
        number++;
    }
</script>

<button {onclick}>
    Click me!
</button>

<h1>{number}</h1>
<p>{number === 0 ? 
"Hey! Why don't you try to click the button" :
`You clicked ${number} times`}</p>
\end{verbatim}

Ma una soluzione più compatta e più nello stile di \texttt{Svelte} consiste
nel registrare lo stato \texttt{number} come \texttt{\$state} e 
creare una variabile derivata \texttt{message} che si aggiorna automaticamente
in base allo stato di \texttt{number}.

\begin{verbatim}
<script lang='ts'>
    let number = $state(0);
    let userInformation = $derived(
        number === 0 ? "Hey! Why don't you try to click the button" :
                       `You clicked ${number} times`
    )
    function onclick() {
        number++;
    }
</script>

<button {onclick}>
    Click me!
</button>

<h1>{number}</h1>
<p>{userInformation}</p>
\end{verbatim}

O ancora possiamo creare una funzione apposita per elaborare il messaggio 
e chiamarla nella callback della derivata, utilizzando {\tt \$derived.by}.

\begin{verbatim}
<script lang='ts'>
    let number = $state(0);
    let userInformation = $derived.by(() => calculateUserInfo(number));
    function onclick() {
        number++;
    }
    function calculateUserInfo(number: number) {
        switch (number) {
            case 0:
                return "Hey! Why don't you try to click the button";
            case 1:
                return "You clicked exactly one time";
            default:
                return `You clicked ${number} times`;
        }
    }
</script>

<button {onclick}>
    Click me!
</button>

<h1>{number}</h1>
<p>{userInformation}</p>
\end{verbatim}

\end{document}
