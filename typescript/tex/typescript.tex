\documentclass[12pt]{article}
\usepackage[margin=1in]{geometry}
\usepackage{amsfonts,amsmath,amssymb}
\usepackage{graphicx}
\usepackage{tcolorbox}
\usepackage{hyperref}
\usepackage{fancyhdr}
\usepackage{float}
\usepackage{enumitem}
\usepackage{titlesec}
\usepackage{lmodern}
\usepackage{xcolor}

% Colori personalizzati
\definecolor{primary}{HTML}{2E86C1} % Blu moderno
\definecolor{secondary}{HTML}{117864} % Verde scuro
\definecolor{accent}{HTML}{D35400} % Arancione acceso
\definecolor{lightgray}{HTML}{F2F3F4} % Grigio chiaro per sfondi

% Impostazioni header e footer
\pagestyle{fancy}
\fancyhf{}
\fancyhead[L]{\textbf{\textcolor{primary}{Typescript}}}
\fancyhead[R]{\textcolor{secondary}{Valerio Molinari}}
\fancyfoot[C]{\thepage}

% Hyperlink
\hypersetup{
    colorlinks=true,
    linkcolor=primary,
    urlcolor=accent,
    citecolor=secondary
}

% Titoli stilizzati
\titleformat{\section}{\color{primary}\normalfont\Large\bfseries}{}{0em}{}
\titleformat{\subsection}{\color{secondary}\normalfont\large\bfseries}{}{0em}{}
\titleformat{\subsubsection}{\color{accent}\normalfont\normalsize\bfseries}{}{0em}{}

% Evidenziazione con box
\newtcolorbox{highlight}{colback=lightgray,colframe=primary!80!black,boxrule=0.5mm,arc=4mm,top=2mm,bottom=2mm,left=4mm,right=4mm}

% Spaziatura e interlinea
\setlength{\parskip}{1em}
\setlength{\parindent}{0pt}
\renewcommand{\baselinestretch}{1.5}

% Nuove definizioni
\newcommand{\important}[1]{\textcolor{accent}{\textbf{#1}}}

\begin{document}

% Copertina
\begin{titlepage}
\begin{center}
\vspace*{3cm}
\Huge\textcolor{primary}{\textbf{Formazione VIM}} \\[1cm]
\Large\textcolor{secondary}{Appunti di Typescript} \\[1cm]
\textcolor{accent}{Valerio Molinari}\\
\vfill
\today
\end{center}
\end{titlepage}

% Indice
\tableofcontents
\newpage

% Esempio di sezione
\section{Introduzione}
Typescript è Javascript con l'aggiunta dei \textbf{tipi}. Tramite lo \textbf{static checking}
controlla gli errori ancora prima di eseguire il codice, in \textbf{fase di sviluppo}.
Typescript è un superset di Javascript, quindi tutto il codice Javascript è anche codice Typescript.

Va considerato che Typescript in realtà compila in Javascript,
quindi il codice Typescript serve al programmatore per evitare di
scrivere bug, al borowser importa ben poco.

\section{Installazione}
Possiamo installare Typescript in un progetto
o in una cartella vuota con il comando:
\begin{highlight}
    \begin{verbatim}
    npm install typescript --save-dev
    \end{verbatim}
\end{highlight}
Questo comando installa Typescript come dipendenza di sviluppo.
Per installare Typescript globalmente, possiamo usare:
\begin{highlight}
    \begin{verbatim}
    npm install -g typescript
    \end{verbatim}
\end{highlight}
Per verificare l'installazione, possiamo usare il comando:
\begin{highlight}
    \begin{verbatim}
    tsc -v
    \end{verbatim}
\end{highlight}

\subsection{Playground}
È anche disponibile un \textbf{\href{https://www.typescriptlang.org/play}{playground}} online per provare il
codice Typescript senza installare nulla.

Oltre a permettere di eseguire del codice senza instanziare
un progetto, il \textit{playground} permette di vedere
esempi dettagliati di codice, permette di configurare
il compilatore e permette di vedere il codice compilato in Javascript
in tempo reale.

\section{Compilazione}
Per compilare il codice Typescript in Javascript, possiamo usare il comando:
\begin{highlight}
    \begin{verbatim}
    tsc nomefile.ts
    \end{verbatim}
\end{highlight}
Questo comando genera un file \textit{.js} con lo stesso nome del file \textit{.ts}.
Possiamo anche compilare tutti i file Typescript in una cartella
con il comando:
\begin{highlight}
    \begin{verbatim}
    tsc
    \end{verbatim}
\end{highlight}
Questo comando cerca un file \textit{tsconfig.json} nella cartella corrente e compila
tutti i file Typescript presenti nella cartella.
Il file \textit{tsconfig.json} è un file di configurazione per il compilatore Typescript.
Possiamo generarlo con il comando:
\begin{highlight}
    \begin{verbatim}
    tsc --init
    \end{verbatim}
\end{highlight}






























\end{document}
