\documentclass[12pt]{article}
\usepackage[margin=1in]{geometry}
\usepackage{amsfonts,amsmath,amssymb}
\usepackage{graphicx}
\usepackage{tcolorbox}
\usepackage{hyperref}
\usepackage{fancyhdr}
\usepackage{float}
\usepackage{enumitem}
\usepackage{titlesec}
\usepackage{lmodern}
\usepackage{xcolor}

% Colori personalizzati
\definecolor{primary}{HTML}{2E86C1} % Blu moderno
\definecolor{secondary}{HTML}{117864} % Verde scuro
\definecolor{accent}{HTML}{D35400} % Arancione acceso
\definecolor{lightgray}{HTML}{F2F3F4} % Grigio chiaro per sfondi

% Impostazioni header e footer
\pagestyle{fancy}
\fancyhf{}
\fancyhead[L]{\textbf{\textcolor{primary}{Rust}}}
\fancyhead[R]{\textcolor{secondary}{Valerio Molinari}}
\fancyfoot[C]{\thepage}

% Hyperlink
\hypersetup{
    colorlinks=true,
    linkcolor=primary,
    urlcolor=accent,
    citecolor=secondary
}

% Titoli stilizzati
\titleformat{\section}{\color{primary}\normalfont\Large\bfseries}{}{0em}{}
\titleformat{\subsection}{\color{secondary}\normalfont\large\bfseries}{}{0em}{}
\titleformat{\subsubsection}{\color{accent}\normalfont\normalsize\bfseries}{}{0em}{}

% Evidenziazione con box
\newtcolorbox{highlight}{colback=lightgray,colframe=primary!80!black,boxrule=0.5mm,arc=4mm,top=2mm,bottom=2mm,left=4mm,right=4mm}

% Spaziatura e interlinea
\setlength{\parskip}{1em}
\setlength{\parindent}{0pt}
\renewcommand{\baselinestretch}{1.5}

% Nuove definizioni
\newcommand{\important}[1]{\textcolor{accent}{\textbf{#1}}}

\begin{document}

% Copertina
\begin{titlepage}
\begin{center}
\vspace*{3cm}
\Huge\textcolor{primary}{\textbf{Formazione VIM}} \\[1cm]
\Large\textcolor{secondary}{Appunti di Rust} \\[1cm]
\textcolor{accent}{Valerio Molinari}\\
\vfill
\today
\end{center}
\end{titlepage}

% Indice
\tableofcontents
\newpage

% Esempio di sezione
\section{Introduzione}
Rust è un linguaggio di programmazione di sistema che è focalizzato 
sulla sicurezza, sulla velocità e sulla concorrenza. 
È progettato per essere \important{sicuro}, \important{concorrente} e 
\important{pratico}.

\section{Installazione}
Per installare Rust su Unix, è sufficiente eseguire il seguente comando:
\begin{verbatim}
curl --proto '=https' --tlsv1.2 -sSf https://sh.rustup.rs | sh
\end{verbatim}
Mentre su Windows è possibile scaricare l'eseguibile dal sito ufficiale.

\section{Hello, World!}
Per creare un programma che stampi a schermo la stringa \texttt{Hello, World!},
è sufficiente creare un file \texttt{main.rs} con il seguente contenuto:
\begin{verbatim}
// main.rs

fn main() {
    println!("Hello, World!");
}
\end{verbatim}
Per compilare ed eseguire il programma, eseguire il comando:
\begin{verbatim}
rustc main.rs && ./main
\end{verbatim}

\section{Progetto}
Cargo è il gestore di pacchetti e di progetti di Rust. 
Per creare un nuovo progetto, possiamo eseguire il comando:
\begin{verbatim}
cargo new <nome_progetto>
\end{verbatim}
Questo comportera la creazione di una cartella con il nome specificato,
contenente un file .toml e una cartella \texttt{src} con il file \texttt{main.rs}.
Per compilare ed eseguire il progetto, ci spostiamo nella cartella del progetto
ed eseguiamo il comando:
\begin{verbatim}
cargo run
\end{verbatim}
Per eseguire il programma compilato senza ricompilarlo, possiamo eseguire il comando:
\begin{verbatim}
./target/debug/<nome_progetto>
\end{verbatim}
Nel caso in cui non fossimo interessati ad eseguirlo, ma solo a compilarlo,
possiamo eseguire il comando:
\begin{verbatim}
cargo build
\end{verbatim}
Infine se volessimo compilare il progetto per la produzione, possiamo eseguire il comando:
\begin{verbatim}
cargo build --release
\end{verbatim}

\end{document}
