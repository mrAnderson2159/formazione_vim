\documentclass[12pt]{article}
\usepackage[margin=1in]{geometry}
\usepackage{amsfonts,amsmath,amssymb}
\usepackage{graphicx}
\usepackage{tcolorbox}
\usepackage{hyperref}
\usepackage{fancyhdr}
\usepackage{float}
\usepackage{enumitem}
\usepackage{titlesec}
\usepackage{lmodern}
\usepackage{xcolor}

% Colori personalizzati
\definecolor{primary}{HTML}{2E86C1} % Blu moderno
\definecolor{secondary}{HTML}{117864} % Verde scuro
\definecolor{accent}{HTML}{D35400} % Arancione acceso
\definecolor{lightgray}{HTML}{F2F3F4} % Grigio chiaro per sfondi

% Impostazioni header e footer
\pagestyle{fancy}
\fancyhf{}
\fancyhead[L]{\textbf{\textcolor{primary}{Rust}}}
\fancyhead[R]{\textcolor{secondary}{Valerio Molinari}}
\fancyfoot[C]{\thepage}

% Hyperlink
\hypersetup{
    colorlinks=true,
    linkcolor=primary,
    urlcolor=accent,
    citecolor=secondary
}

% Titoli stilizzati
\titleformat{\section}{\color{primary}\normalfont\Large\bfseries}{}{0em}{}
\titleformat{\subsection}{\color{secondary}\normalfont\large\bfseries}{}{0em}{}
\titleformat{\subsubsection}{\color{accent}\normalfont\normalsize\bfseries}{}{0em}{}

% Evidenziazione con box
\newtcolorbox{highlight}{colback=lightgray,colframe=primary!80!black,boxrule=0.5mm,arc=4mm,top=2mm,bottom=2mm,left=4mm,right=4mm}

% Spaziatura e interlinea
\setlength{\parskip}{1em}
\setlength{\parindent}{0pt}
\renewcommand{\baselinestretch}{1.5}

% Nuove definizioni
\newcommand{\important}[1]{\textcolor{accent}{\textbf{#1}}}

\begin{document}

% Copertina
\begin{titlepage}
\begin{center}
\vspace*{3cm}
\Huge\textcolor{primary}{\textbf{Formazione VIM}} \\[1cm]
\Large\textcolor{secondary}{Appunti di Rust} \\[1cm]
\textcolor{accent}{Valerio Molinari}\\
\vfill
\today
\end{center}
\end{titlepage}

% Indice
\tableofcontents
\newpage

% Esempio di sezione
\section{Introduzione}
Rust è un linguaggio di programmazione di sistema che è focalizzato 
sulla sicurezza, sulla velocità e sulla concorrenza. 
È progettato per essere \important{sicuro}, \important{concorrente} e 
\important{pratico}.

\section{Installazione}
Per installare Rust su Unix, è sufficiente eseguire il seguente comando:
\begin{verbatim}
curl --proto '=https' --tlsv1.2 -sSf https://sh.rustup.rs | sh
\end{verbatim}
Mentre su Windows è possibile scaricare l'eseguibile dal sito ufficiale.

\section{Rust-analyzer}
Rust-analyzer è un'estensione per Visual Studio Code che permette di
avere un'esperienza di programmazione in Rust più fluida. Per installarlo,
è sufficiente cercare l'estensione nel marketplace di Visual Studio Code.

Per farlo funzionare invece è necessario o aprire un progetto Rust direttamente in VSC,
oppure bisogna modificare il file \texttt{settings.json} aggiungendo la seguente riga:
\begin{verbatim}
    "rust-analyzer.linkedProjects": [
        PATH_DEL_PROGETTO
    ],
\end{verbatim}

\section{Hello, World!}
Per creare un programma che stampi a schermo la stringa \texttt{Hello, World!},
è sufficiente creare un file \texttt{main.rs} con il seguente contenuto:
\begin{verbatim}
// main.rs

fn main() {
    println!("Hello, World!");
}
\end{verbatim}
Per compilare ed eseguire il programma, eseguire il comando:
\begin{verbatim}
rustc main.rs && ./main
\end{verbatim}

\section{Progetto}
Cargo è il gestore di pacchetti e di progetti di Rust. 
Per creare un nuovo progetto, possiamo eseguire il comando:
\begin{verbatim}
cargo new <nome_progetto>
\end{verbatim}
Questo comportera la creazione di una cartella con il nome specificato,
contenente un file .toml e una cartella \texttt{src} con il file \texttt{main.rs}.
Per compilare ed eseguire il progetto, ci spostiamo nella cartella del progetto
ed eseguiamo il comando:
\begin{verbatim}
cargo run
\end{verbatim}
Per eseguire il programma compilato senza ricompilarlo, possiamo eseguire il comando:
\begin{verbatim}
./target/debug/<nome_progetto>
\end{verbatim}
Nel caso in cui non fossimo interessati ad eseguirlo, ma solo a compilarlo,
possiamo eseguire il comando:
\begin{verbatim}
cargo build
\end{verbatim}
Infine se volessimo compilare il progetto per la produzione, possiamo eseguire il comando:
\begin{verbatim}
cargo build --release
\end{verbatim}

\pagebreak
\section{Variabili}
In rust le variabili sono tipate e si possono dichiarare in due modi:
\begin{verbatim}
let x = 5;
let y: i32 = 10;
\end{verbatim}
Di default esse sono immutabili, pertanto, per dichiarare una 
variabile mutabile, è necessario utilizzare la keyword \texttt{mut}:
\begin{verbatim}
let mut x = 5;
x = 10;
\end{verbatim}
Per quanto riguarda le costanti è convenzione scriverle in maiuscolo
in sneak case:
\begin{verbatim}
const MAX_POINTS: u32 = 100_000;
\end{verbatim}

\section{Tipi di dato}
\subsection{Primitivi}
Rust è un linguaggio staticamente tipato, il che significa che
deve sapere i tipi di tutte le variabili a tempo di compilazione.
I tipi di dato primitivi sono i seguenti:
\begin{itemize}
    \item \texttt{bool}: booleano
    \item \texttt{char}: carattere
    \item \texttt{i8}, \texttt{i16}, \texttt{i32}, \texttt{i64}, \texttt{i128}: interi con segno
    \item \texttt{u8}, \texttt{u16}, \texttt{u32}, \texttt{u64}, \texttt{u128}: interi senza segno
    \item \texttt{f32}, \texttt{f64}: numeri in virgola mobile
\end{itemize}

\subsection{Composti}
I tipi di dato composti sono i seguenti:
\begin{itemize}
    \item \texttt{string}: sequenza di caratteri
    \item \texttt{array}: collezione di valori dello stesso tipo
    \item \texttt{vector}: collezione di valori dello stesso tipo di lunghezza variabile
    \item \texttt{tuple}: collezione di valori di tipi diversi    
\end{itemize}

\subsubsection{Stringhe}
Le stringhe a seconda di come vengono dichiarate possono essere
fisse o variabili. Per dichiarare una stringa fissa, dichiariamo
un "puntatore" a stringa, utilizzando il tipo \texttt{\&str} e le virgolette.
Per le stringhe variabili utiliziamo il tipo \texttt{String} e 
il metodo \texttt{String::from}:
\begin{verbatim}
let s1: &str = "Hello, World!";
let s2: String = String::from("Hello, World!");
\end{verbatim}



\section{Ownership}
Rust gestisce la memoria in modo molto particolare, utilizzando il concetto
di ownership. Ogni valore in Rust ha una variabile che è la sua proprietaria.
Quando la proprietà esce dallo scope, il valore viene deallocato.

Pensiamo, per esempio, ad una stringa. La stringa è in parte allocata nello stack
e in parte nello heap. Nello heap è allocata la sequenza di caratteri,
mentre nello stack abbiamo un puntatore a tale sequenza, la lunghezza della stringa
e la capacità della stringa. Se assegnamo la stringa ad un'altra variabile
\begin{verbatim}
let s1 = String::from("Hello, World!");
let s2 = s1;
\end{verbatim}
il puntatore a stringa viene copiato, ma la proprietà viene trasferita a \texttt{s2}.
Di conseguenza, se proviamo ad accedere a \texttt{s1} otterremo un errore.


\section{Borrowing}
Per risolvere il problema dell'ownership, possiamo utilizzare il concetto di borrowing.
Il borrowing ci permette di prendere in prestito un riferimento ad una variabile, 
senza trasferirne la proprietà. Possiamo farlo in due modi:
\begin{itemize}
    \item \textbf{Borrowing immutabile}: possiamo prendere in prestito un riferimento
    ad una variabile in modo tale da poterla solamente leggere, ma non modificarla.
    \begin{verbatim}
    let s1 = String::from("Hello, World!");
    let s2 = &s1;
    \end{verbatim}
    \item \textbf{Borrowing mutabile}: possiamo prendere in prestito un riferimento
    ad una variabile in modo tale da poterla modificare.
    \begin{verbatim}
    let mut s1 = String::from("Hello, World!");
    let s2 = &mut s1;
    \end{verbatim}
\end{itemize}

\subsection{Regole di borrowing}
Le regole di borrowing sono le seguenti:    
\begin{itemize}
    \item Possiamo avere un solo riferimento mutabile ad una variabile alla volta.
    \item Possiamo avere più riferimenti immutabili ad una variabile.
    \item Non possiamo avere un riferimento mutabile e immutabile ad una variabile.
\end{itemize}


































































\end{document}